\documentclass[letterpaper, oneside, 12pt]{article}
\usepackage{graphicx}
\usepackage{rotating}
\usepackage[american]{babel}

\selectlanguage{american}

\newcommand{\ie}{i.e.,}
\newcommand{\eg}{e.g.,}


% document
\begin{document}

%\begin{titlepage}
\hspace{-7mm}% put the page to the perfect position
\begin{minipage}{\textwidth}
\begin{center}
\vspace{.5cm}
{\large \bf Project Outline\\[3ex]}
{\huge \bf Project 'BCI2000'}
\\[1.5cm]
{\Large Gerwin Schalk\\}
{\Large Thilo Hinterberger\\}
{\Large Dennis J. McFarland\\[1.5cm]}
%
\begin{minipage}{13cm}
  \begin{minipage}[c]{13cm}
    \begin{center}
      {\Large \bf New York State Department of Health\\[2ex]}
      {\large \bf Wadsworth Center\\[0.5ex]
       Laboratory of Nervous Systems Disorders\\[4ex]}
      {\Large \bf Eberhard--Karls--Universit\"at T\"ubingen\\[2ex]}
      {\large \bf Medizinische Fakult\"at\\[0.5ex]
       Institut f\"ur Medizinische Psychologie\\[0.5ex]}
    \end{center}
  \end{minipage}
  \\[1.0cm]
  \begin{minipage}[c]{6cm}
    \centerline{\includegraphics{figures/DOHlogo}}
  \end{minipage}
  \hspace{1.5cm}
  \begin{minipage}[c]{3cm}
    \centerline{\includegraphics{figures/EKUlogo}}
  \end{minipage}
\end{minipage}
%
\\[0.5cm]
\textbf{Sponsors} \\
\textit{Jonathan R. Wolpaw and Niels Birbaumer}
\\[1.0cm]
{Albany, NY} \\[1ex]
{February 2000--May 2001}
\\[1ex]Updated May 2003, J\"urgen Mellinger
\end{center}
\end{minipage}
\end{titlepage}

\title{BCI2000 g.MOBIlab Support}
\author{Gerwin Schalk\\ \small{2005 Brain-Computer Interface Research and Development Program}\\ \small{Wadsworth Center, New York State Department of Health}}
\maketitle
\centerline{\includegraphics[height=2.4cm,keepaspectratio=true]{BCI2000logo}}

%\tableofcontents

%\newpage

%\begin{abstract}
%\end{abstract}

\section{Introduction}

\sloppypar \emph{g.MOBIlab} is an amplifier/digitizer combination from g.tec 
medical engineering GmbH / Guger Technologies OEG (\texttt{http://www.gtec.at}). 
This document describes support for this device in BCI2000, which consists of a 
BCI2000-compatible Source Module (\texttt{gMOBIlab.exe}). 

\section{g.MOBIlab Hardware}

The MOBIlab device supports 8 analog input channels digitized at 16 bit 
resolution and sampled at a fixed 256 Hz sampling rate. In its standard 
configuration, channels 1-2 have a sensitivity of $\pm100\mu V$, channels 3-4 
have a sensitivity of $\pm500\mu V$, channels 5-6 have a sensitivity of 
$\pm5mV$, and channels 7-8 have a sensitivity of $\pm5V$. The input range of the 
A/D converter is approximately equal to this sensitivity and thus, for example, 
one LSB for channel 1 or 2 is roughly $\frac{200\mu V}{65535}=0.003\mu V$. However,
the actual input range of the A/D converter is slightly larger than the sensitivity
of each channel (so that the A/D converter can detect when the amp saturates), and
thus, exact LSB values have to be determined for each channel using a calibration signal.

This device only has one A/D converter and 
thus samples are digitized at slightly different times. BCI2000 has a feature 
that can align samples in time (parameter \emph{AlignChannels} in Section 
\emph{Filtering}), which needs to be turned on (i.e., \emph{AlignChannels} needs 
to be 1).

An additional feature of the MOBIlab is 2 digital input/output lines. The MOBIlab source module is configured such that channel 9 corresponds to the value of the digital lines, 
which are configured as input lines.

\section{g.MOBIlab Source Module}

The BCI2000-compatible Source Module \texttt{gMOBIlab.exe} can be used instead of 
any other source module. In addition to standard parameters (\ie{} 
\emph{SampleBlockSize}, \emph{SamplingRate}, \emph{SoftwareCh}, 
\emph{TransmitCh}, \emph{TransmitChList}), it only contains one
parameter (\emph{COMport}):

\begin{description}
 \item [COMport]           Serial port of the attached MOBIlab device, e.g., COM2: 
 \item [SampleBlockSize]   Samples per digitized block. A value of 8 corresponds to
                           a BCI2000 system rate of 32 Hz ($\frac{8 samples}{256 Hz}$).
 \item [SamplingRate]      The sampling rate of the MOBIlab. This value has to be 256.
 \item [SoftwareCh]        The total number of channels. This number can be 1 to 9.
                                      If it is set to 9, then channels 1-8 represent 8 analog input
                                        channels, and channel 9 represents the values of the two 
                                       digital lines.
 \item [TransmitCh]        The number of channels that are transmitted to the BCI2000 
                           Signal Processing module. See the BCI2000 Project
                           Outline for further information.
 \item [TransmitChList]    The list of channels that are transmitted to the BCI2000 
                           Signal Processing module. See the BCI2000 Project
                           Outline for further information.                                                                        
\end{description}




\end{document}

