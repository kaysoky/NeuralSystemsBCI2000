\documentclass[letterpaper,oneside,12pt]{article}
%\documentclass[letterpaper,oneside,12pt]{report}
\usepackage{graphicx}
\usepackage{rotating}
\usepackage[american]{babel}

\selectlanguage{american}

% page layout
%  \setlength{\hoffset}{-1in} 
  %\setlength{\voffset}{-1in}
%  \setlength{\textwidth}{15cm}
%  \setlength{\oddsidemargin}{3.5cm}
%  \setlength{\evensidemargin}{0.2cm}
%  %\setlength{\topmargin}{2.5cm}
%  \setlength{\headsep}{5ex}
%  \setlength{\textheight}{22cm}
%  \renewcommand{\baselinestretch}{1.05}
%  \raggedbottom
%  \newlength{\pictheight}
%  \setlength{\pictheight}{\textheight}
%  \addtolength{\pictheight}{-2cm}

% document
\begin{document}

%\begin{titlepage}
\hspace{-7mm}% put the page to the perfect position
\begin{minipage}{\textwidth}
\begin{center}
\vspace{.5cm}
{\large \bf Project Outline\\[3ex]}
{\huge \bf Project 'BCI2000'}
\\[1.5cm]
{\Large Gerwin Schalk\\}
{\Large Thilo Hinterberger\\}
{\Large Dennis J. McFarland\\[1.5cm]}
%
\begin{minipage}{13cm}
  \begin{minipage}[c]{13cm}
    \begin{center}
      {\Large \bf New York State Department of Health\\[2ex]}
      {\large \bf Wadsworth Center\\[0.5ex]
       Laboratory of Nervous Systems Disorders\\[4ex]}
      {\Large \bf Eberhard--Karls--Universit\"at T\"ubingen\\[2ex]}
      {\large \bf Medizinische Fakult\"at\\[0.5ex]
       Institut f\"ur Medizinische Psychologie\\[0.5ex]}
    \end{center}
  \end{minipage}
  \\[1.0cm]
  \begin{minipage}[c]{6cm}
    \centerline{\includegraphics{figures/DOHlogo}}
  \end{minipage}
  \hspace{1.5cm}
  \begin{minipage}[c]{3cm}
    \centerline{\includegraphics{figures/EKUlogo}}
  \end{minipage}
\end{minipage}
%
\\[0.5cm]
\textbf{Sponsors} \\
\textit{Jonathan R. Wolpaw and Niels Birbaumer}
\\[1.0cm]
{Albany, NY} \\[1ex]
{February 2000--May 2001}
\\[1ex]Updated May 2003, J\"urgen Mellinger
\end{center}
\end{minipage}
\end{titlepage}

\title{BCI2000-Compatible Audio-Visual P3 Task}
\author{J\"{u}rgen Mellinger, Gerwin Schalk}
\maketitle

\tableofcontents

\newpage 

\begin{abstract}

The main purpose of this task is to present a series of (auditory or visual) 
stimuli sequentially to the user of the BCI system. The sequence and nature of 
the stimuli can be defined by the investigator. In addition to stimulus 
delivery, the task can optionally be used in conjunction with BCI2000's P300 
Signal Processing module (P3SignalProcessing.exe) to provide feedback to a 
selected stimulus in either a copy or a free mode.

\end{abstract}

\section{Functionality}

\subsection{Stimulus Definition}

Stimuli are defined in a Windows\texttrademark{} ini-file. This file defines the 
total number of stimuli, as well as the details of each stimulus. Each 
stimulus is defined by the following properties:
\begin{enumerate}
 \item Caption
 \item Icon file
 \item Audio file
\end{enumerate}

In addition to stimuli, the file contains definition for a stimulus that 
announces what to focus on, and a stimulus that announces the result. These
stimuli are only used when the task is set to copy or free mode.

The following is an example of a stimulus definition file with two stimuli:
\begin{verbatim}
[global]
numstimuli      = 2

[focuson]
caption         = Please focus on
icon            = icons\focus on.bmp
audio           = wav\focus on.wav

[result]
caption         = The result was
icon            = icons\result.bmp
audio           = wav\result.wav

[stimulus1]
caption         = Donkey
icon            = icons\donkey.bmp
audio           = wav\snicker.wav

[stimulus2]
caption         =
icon            = icons\elefant.bmp
audio           = wav\trumpet.wav
\end{verbatim}

\emph{Comments:} The stimulus properties might contain white spaces. A 
caption/icon/audio file are not being presented, if they are not defined (e.g., 
see caption in stimulus2). The stimulus definition file does \emph{not} contain 
a description on how the stimuli are presented. Stimulus numbers start at 1. For 
each run, the ini-file will be copied to the same location as the actual data 
file. For example, if the data file is "JJS007R03.dat," and the ini-file is 
"stimdefs.ini," then the ini-file will be copied and named "JJS007R03.ini."

\subsection{Stimulus Sequence}

Stimuli are presented in a certain sequence. This sequence can either be 
deterministic, i.e., defined by the investigator, or random. 

\subsubsection{Deterministic Sequence:}The investigator defines the order by 
entering a list of stimulus IDs to be presented. As an example: 
\begin{verbatim} 
1 5 3 4 2 
\end{verbatim} 
defines a sequence in which stimulus 1 is first presented, followed by 
stimulus 5, etc.

\subsubsection{Random Sequence:}The investigator defines the stimulus 
probabilities for each stimulus. Stimulus probabilities do not have to
add up to 100\%. (The sum of all probabilities will define 100\%).
As an example: 
\begin{verbatim} 
60 20 20
\end{verbatim} 
defines a probability of 60\% for stimulus 1, and 20\% each for stimuli 2 and 3.

A sequence can be played multiple times. The investigator can define how many 
times sequences are played.

\subsection{Stimulus Delivery}

For any stimulus, delivery occurs simultaneously for caption\footnote{A caption, 
if defined, always appears in front on an icon.}, icon, and audio. (A computer 
can only execute commands in sequence, but the time difference between start of 
presentation of caption, icon, and audio, is negligible). \emph{Comment:} A 
knowledgeable investigator has to understand the implications of audio files 
that are of unequal length !

An investigator can specify:
\begin{itemize}
 \item Size and position of the target window (using the same scheme/parameters 
       as used by the RJB task, Oddball paradigm, or P3 speller).
 \item Width and height of caption and icon in percent of screen width/height
 \item Whether captions, icons, or audio files will be presented
       (i.e., a global switch -- for each stimulus, captions/icons/wave files
       will not be presented individually, if not defined)
 \item The volume for audio playback
 \item Window background color in RGB
 \item Caption color in RGB
 \item The duration during which a stimulus is presented (in units of SampleBlocks)
 \item The inter-stimulus interval that follows completed stimulus presentation \footnote{During 
       the inter-stimulus interval, the screen is blank and audio is turned off.}
 \item A minimum and maximum time (in units of SampleBlocks) that will be added randomly
       to the inter-stimulus interval. If these variables are both set to 0,
       the actual inter-stimulus interval will always be exactly as defined above.
       If these variables are set to, for example, 0 and 3, inter-stimulus intervals
       will randomly be longer by between 0 and 3 units.
 \item A Comment. A user can enter comments to the specific run in a string parameter.
\end{itemize}


\section{Processing of Classification Results}

The task can be configured to interpret results communicated to it by the P3 
Signal Processing module. These results represent a judgment on which of the 
stimuli was most likely selected. Handling of these results is identical to the 
P3 Spelling Task. 

When it transmits a classification result, Signal Processing sets the state 
\emph{StimulusCodeRes} to the stimulus code that was originally transmitted to 
it by the user application. For example, when it sets \emph{StimulusCodeRes} to 
3, it indicates that it transmits classification results for stimulus 3. In 
addition, it sets \emph{StimulusTypeRes} to reflect the type of the stimulus 
(0=non-target, 1=target) when the system is in copy mode. Signal Processing 
transmits the classication result as one number (i.e., the first control signal).

\subsection{Free Mode}

The task can be configured to operate in free mode. In this case, the sequence 
of stimulus delivery is followed by a time period, in which the result of Signal 
Processing's classification is announced. The final classification result is
the stimulus with the highest classification result.

In order to deliver this announcement, the system uses the stimulus defined in 
the \texttt{[result]} section of the stimulus ini-file. This announcement is 
followed by delivery of the determined stimulus. In other words, after a 
sequence of stimulus delivery, the system might play a .wav file that says: "the 
result is," followed by a .wav file that says "yes." (assuming "yes" represents
the stimulus that produced the highest classification result).

Finally, the task sends this result to the operator module as an ASCII text 
message so that it appears in a log window.

Free mode does not terminate until the investigator suspends operation.


\subsection{Copy Mode}

Copy mode is similar to free mode. In copy mode, the investigator can define a 
list of stimuli to be copied (e.g., "3 5 4"). In this example, the user has to 
attend to stimulus 3 for the first sequence, 5 for the second sequence, etc.

In addition to an announcement of the result, in copy mode the delivery of 
stimuli is preceded by an announcement that describes which stimulus the user
has to attend to. This announcement uses the stimulus that is defined in the
\texttt{[focuson]} section of the ini-file. This announcement is followed by
delivery of the desired target stimulus. As an example, the system might say
"Please focus now on" ... "yes," before it starts with the sequence of
stimulus delivery.

Copy mode terminates (i.e., the task suspends) when the user finished copying 
all stimuli specified by the investigator.

\section{Parameters}

Fill in the parameters incl. description, once completed.


\section{States}

The time line of stimulus delivery is encoded in state variables as defined in
Table \ref{tab:states}.
\begin{table}
\begin{center}
\begin{tabular}[ht]{|l|l|l|}
\hline
\bf{State Name}& \bf{Bits}            & \bf{Description} \\
\hline
\hline
 SelectedStimulus & 7 & stimulus ID of finally selected stimulus \\
                  &   & only $>$0 when determining selected stimulus\\
\hline
 PhaseInSequence  & 2 & 0 in inter-stimulus interval\\
                  &   & 1 period prior to stimulus sequence (if any)\\
                  &   & 2 stimulus sequence period\\
                  &   & 3 period after stimulus sequence (if any)\\
\hline
 StimulusTime     & 16& time in ms as calculated in P3 Speller\\
\hline
 StimulusCode     & 7 & stimulus ID of currently visible stimulus\\
                  &   & or 0 if no stimulus visible\\
\hline
 StimulusType     & 1 & 0 in free mode; copy mode :\\
                  &   & 1 when current stimulus equals target stimulus\\
                  &   & 0 otherwise\\
\hline
 Flashing         & 1 & 1 during stimulus presentation, 0 otherwise\\
\hline
\end{tabular}
\caption{Encoding scheme for this task.}
\label{tab:states}
\end{center}
\end{table}

%\appendix

\end{document}

