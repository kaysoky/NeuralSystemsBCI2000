\chapter{Introduction}

\section{Purpose}

All presently available augmentative communication systems depend in some 
measure on voluntary muscle control.  Thus, they are useless to those who are 
totally paralyzed and to some others with severe motor disabilities.  EEG--based 
communication, because it does not depend on voluntary muscle control, could 
provide a valuable new communication and control option for these individuals.  
Over the past decade, a number of laboratories have begun developing EEG--based 
Brain Computer Interfaces (i.e., BCIs) as a new augmentative technology for 
people with motor disabilities.

The BCI2000 standard (as described in the BCI2000 Project Outline) has been 
designed in a cooperation between the Laboratory of Nervous Systems Disorders at 
the Wadsworth Center in the New York State Department of Health and the Institut 
f\"ur Medizinische Psychologie at the Medizinische Fakult\"at at the 
Eberhard--Karls--Universit\"at in T\"ubingen/Germany, in an effort to create a 
well documented and open system that is open for extensions; this document 
describes one particular implementation of this standard.

Not only does this document describe the software already in place, it is also 
intended to enforce compatibility of future modifications or add--ons.

\section{Scope}

This document is intended to give a detailed technical description of the 
BCI2000 software project. It does not, however, explain the BCI2000 standard 
itself, or the rationale behind the implementation or standard.

\section{Intended Audience}

The intended audience for this document are engineers or researchers, who want 
to modify and/or extend the existing reference implementation. As described 
software is implemented using Borland's C++ Builder, the reader should have some 
knowledge of the C/C++ programming language.

\section{List of System Components}

The sofware package consists of four Win32 executables:

\begin{table}[ht]
 \centering
 \begin{tabular}{|l|l|l|}
  \hline
  \textbf{Module Name} & \textbf{Filename} & \textbf{Current Version} \\
  \hline
  Operator & Operat.exe & V0.20 \\
  \hline
  EEG source & EEGsource.exe & V0.20 \\
  \hline
  Signal Processing & SignalProcessing.exe & V0.20 \\
  \hline
  Application & Application.exe & V0.20 \\
  \hline
 \end{tabular}
 \caption{The four executables}
\end{table}   


\section{References}

The BCI2000 project homepage contains all relevant documentation, source code, 
and additional analysis tools:\\ 
http://www.bciresearch.org/BCI2000/bci2000.html

\section{Content Summary}

This document presents an overview of the system, the design considerations 
leading to the system architecture, describes the system architecture itself, 
and finally details the system design.
