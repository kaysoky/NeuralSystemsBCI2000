\chapter{Architectural Strategies}


%Describe any design decisions and/or strategies that affect the overall 
%organization of the system and its higher-level structures. These strategies 
%should provide insight into the key abstractions and mechanisms used in the 
%system architecture. Describe the reasoning employed for each decision and/or 
%strategy (possibly referring to previously stated design goals and principles) 
%and how any design goals or priorities were balanced or traded-off. Such 
%decisions might concern (but are not limited to) things like the following: 

%     Use of a particular type of product (programming language, database, library, etc. ...) 
%     Reuse of existing software components to implement various parts/features of the system 
%     Future plans for extending or enhancing the software 
%     User interface paradigms (or system input and output models) 
%     Hardware and/or software interface paradigms 
%     Error detection and recovery 
%     Memory management policies 
%     External databases and/or data storage management and persistence 
%     Distributed data or control over a network 
%     Generalized approaches to control 
%     Concurrency and synchronization 
%     Communication mechanisms 
%     Management of other resources 

%Each significant strategy employed should probably be discussed in its own 
%subsection, or (if it is complex enough) in a separate design document (with an 
%appropriate reference here of course). Make sure that when describing a design 
%decision that you also discuss any other significant alternatives that were 
%considered, and your reasons for rejecting them (as well as your reasons for 
%accepting the alternative you finally chose). Sometimes it may be most effective 
%to employ the "pattern format" for describing a strategy. 

